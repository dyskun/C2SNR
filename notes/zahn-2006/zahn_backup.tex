\documentclass[12pt]{article}
\usepackage[a4paper,left=2.5cm,right=2.5cm,top=2.5cm,bottom=2.5cm]{geometry}
%\usepackage[margin=0.5in]{geometry}
\usepackage[utf8]{inputenc}
\usepackage{graphicx}
\usepackage{float}
\usepackage{multicol}
\usepackage{wrapfig}
\usepackage{multirow}
\usepackage{subcaption}
\graphicspath{ {./images/} }
\usepackage{amssymb}
\usepackage{amsmath}
\usepackage{graphicx}
\usepackage{epstopdf}
\usepackage{bm}
\usepackage{mdframed}

\usepackage{cite}

%\usepackage[
%backend=biber,
%style=authoryear,
%sorting=ynt
%]{biblatex}
%\addbibresource{project_log.bib}


\usepackage{hyperref}
\hypersetup{
    colorlinks=true,
    linkcolor=blue,
    filecolor=magenta,      
    urlcolor=cyan,
}
\urlstyle{same}

\newcommand{\bma}{\begin{math}}
\newcommand{\ema}{\end{math}}
\newcommand{\beq}{\begin{equation}}
\newcommand{\eeq}{\end{equation}}
\newcommand{\beqa}{\begin{eqnarray}}
\newcommand{\eeqa}{\end{eqnarray}}
\newcommand{\beqal}{\begin{aligned}}
\newcommand{\eeqal}{\end{aligned}}
\newcommand{\bc}{\begin{center}}
\newcommand{\ec}{\end{center}}
\newcommand{\bit}{\begin{itemize}}
\newcommand{\eit}{\end{itemize}}


\def\n{{\bf  \hat n}}
\def\k{{\bf k}}
\def\l{{\bf l}}
\def\L{{\bf L}}
\def\r{{\bf r}}
\def\x{{\bf x}}
\def\u{{\bf u}}
\def\F{{\bf F}}
\def\K{{\rm K}}
\def\mK{{\rm mK}}
\def\arcmin{{\rm arcmin}}
\def\ln{{\rm ln}}
\def\max{{\rm max}}
\def\min{{\rm min}}
\def\tot{{\rm tot}}
\def\iul{{\rm I}}
\def\il{{\tilde{\rm I}}}
\def\itot{{\tilde{\rm I}^{\rm tot}}}
\def\pul{{\rm P}}
\def\pl{{\tilde{\rm P}}}
\def\ptot{{\tilde{\rm P}^{\rm tot}}}
\def\cul{{\rm C}}
\def\cl{{\tilde{\rm C}}}
\def\ctot{{\tilde{\rm C}^{\rm tot}}}
\def\km{{\rm km}}
\def\kg{{\rm kg}}
\def\h{{\rm h}}
\def\erf{{\rm erf}}
\def\erfc{{\rm erfc}}
\def\mpch{{\rm Mpc/h}}
\def\mhz{{\rm MHz}}
\def\d2l{\frac{d^2l}{(2\pi)^2}}
\def\dko{\frac{dk_1}{2\pi}}
\def\dkt{\frac{dk_2}{2\pi}}
\def\dtheta{\delta \theta}
\def\dang{\mathcal{D}}



\numberwithin{equation}{section}
\begin{document}
\tableofcontents
\pagebreak

\section{Extending Hu's quadratic estimator to three dimensions}
\paragraph{Objective} To understand eq. 22 to 31 in Zahn and Zaldarriaga 2006

\subsection{Introduction}
\begin{itemize}
	\item Unlike CMB, we can use information from multiple planes in the case of 21 cm signal.
	\item But the different planes would be correlated. No simple way to take this correlation into account while constructing the estimator.
	\item Instead, we divide the 3D temperature fluctuations into radial and transverse parts.
\end{itemize}

\subsection{Notation}
We use the terminology and notation as described in \href{http://arxiv.org/abs/astro-ph/9905116v4}{Hogg 2000 arXiv:9905116}.
\paragraph{Observed volume}
\begin{itemize}
	\item We define the observed volume as
		\begin{itemize}
			\item located at redshift $ z $. 
			\item having a width of solid angle $ d\Omega $ along the transverse direction as seen by the observer
			\item having the depth along the line of sight $ D_C $ = comoving distance between redshift $ z$ and $ z' $, where $ z < z' $.
		\end{itemize}
	\item The  distance between the observer and the volume is $ D_M $. It is called transverse comoving distance as per the definition in Hogg 2000.
	\item The transverse comoving size of the volume is $ S $. 
\end{itemize}

\paragraph{Intensity field}
Field $ I(\textbf{r}) $ at position $ \textbf{r} $ in comoving space
\beq
\beqal
I(\textbf{r}) &= \int \frac{d^3\kappa}{(2\pi)^3} I(\kappa) e^{i \kappa \cdot r}
\\
&= \int \frac{d^2l}{(2\pi)^2} \int \frac{dk_{\parallel}}{2\pi} \frac{I(k_\perp, k_\parallel)}{D_M^2} e^{i(l\cdot \theta + k_\parallel x_\parallel)}
\eeqal
\eeq
where \textbf{k} is the wave vector.

\subsection{Formalism}
We divide \textbf{k} into radial (parallel) and transverse components: $ k_\parallel $ and $ k_\perp $. Corresponding to this, we also have $ x_\parallel $ and $ x_\perp $.

\paragraph{Discretize $k_\perp$} Using
\beqa
\theta = \frac{S}{D_M}
\eeqa

and 
\beqa
l = \frac{2 \pi}{\theta}
\eeqa

we get 

\beq
\beqal
\kappa_\perp &= \frac{2 \pi}{S}
\\
&= \frac{2 \pi }{\theta D_M}
\\
&= \frac{l}{D_M}
\eeqal
\eeq

\paragraph{Discretize $ k_\parallel $}Discretizing $ D_C $ into many slices along the line of sight. $ j $ is the discretizing factor.

\beqa
k_\parallel  = j\frac{2\pi}{D_C}; \delta(k_\parallel - k_\parallel') = \frac{D_C}{2\pi} \delta_{j_1, j_2} 
\eeqa
where $ j $ represents the $ j^{th} $ element of the volume.

\paragraph{Final expression of Intensity with discrete \textbf{k}}
\beqa
I(r) = \int\frac{d^2l}{(2\pi)^2} \sum_j \left(\frac{I(k_\perp, k_\parallel)}{D_M^2 D_C} \right) e^{i \textbf{k} \cdot \textbf{r}}
\eeqa

\paragraph{Simplify}
Define
\beqa
\hat{I} (l, k_\parallel) \equiv \frac{I(k_\perp, k_\parallel)}{D_M^2D_C}
\eeqa

Starting with
\beq
\beqal
I(\kappa_\perp,\kappa_\parallel ; \tau) &= \int dx_\parallel \int d^2\vec{x}_\perp e^{-i \kappa_\parallel \cdot x_\parallel} e^{-i \kappa_\perp \cdot x_\perp} I(\vec{x}_\perp, x_\parallel; \tau) \label{eq:foreman1}
\eeqal
\eeq

\beq
\beqal
\langle I(k_\perp, k_\parallel) I^*(k_\perp', k_\parallel')   \rangle &= \delta(\textbf{k} - \textbf{k}') (2\pi)^3 P(k_\perp, k_\parallel)
\\
&= (2\pi)^2 \delta(l - l') D_M^2 (2\pi) \delta(k_\parallel - k'_\parallel) P(k_\perp, k_\parallel) \label{eq:3dcorr}
\eeqal
\eeq
where  $ P(k_\perp, k_\parallel) $ is the 3D power spectrum of the intensity field, we get
%Then we get the equation 2.1 of \textbf{Foreman et. al. 2018}
%\beqa
%\hat{I}(l, k_\parallel, \tau) = \frac{1}{D_C D_M^2} I(k_\perp, k_\parallel; \tau) \equiv \frac{1}{D_C D_M^2} I(l/D_M, k_\parallel; \tau)
%\eeqa
\beqa
%\langle I(k_\perp, k_\parallel) I^*(k_\perp', k_\parallel')   \rangle &= (2\pi)^2 \delta(l - l') D_M^2 (2\pi) \delta(k_\parallel - k'_\parallel) P(k_\perp, k_\parallel)
%\\
\frac{\langle I(k_\perp, k_\parallel) I^*(k_\perp', k_\parallel')   \rangle}{(D_C D_M^2)^2} &=& \frac{(2\pi)^2 \delta(l - l') D_M^2 (2\pi) \delta(k_\parallel - k'_\parallel) P(k_\perp, k_\parallel) }{(D_C D_M^2)^2}
\\
\langle \hat{I}(l, k_\parallel) \hat{I}^*(l', k_\parallel')   \rangle &=& \frac{(2\pi)^2 \delta(l - l')  (2\pi) \delta(k_\parallel - k'_\parallel) P(k_\perp, k_\parallel) }{(D_C^2 D_M^2)}
\\
&=& \frac{(2\pi)^2 \delta(l - l')  (2\pi) \delta(j - j') (D_C/2\pi) P(k_\perp, k_\parallel) }{(D_C^2 D_M^2)}
\\
\langle \hat{I}(l, j \frac{2\pi}{D_C}) \hat{I}^*(l', j'\frac{2\pi}{D_C})   \rangle  &=& (2\pi)^2 \delta(l - l') \delta(j - j')  \frac{P(k_\perp, j\frac{2\pi}{D_C}) }{(D_C D_M^2)}
\eeqa
This is equation 27 in ZZ2006.

\paragraph{Definition of $ C_{l,j} $}
We now define the angular power spectrum for the seperate values of j, 
\beqa
C_{l,j} \equiv \frac{P \left( \sqrt{\frac{l^2}{D_M^2} + (j\frac{2\pi}{D_C})^2} \right)}{D_C D_M^2}
\eeqa
where $ P $ now represents the spherically averaged power spectrum.


Now we are going to derive Equation 31 (Equation A21) of ZZ2006.

\section{Lensing reconstruction noise}
\subsection{Mathematical form of a quadratic estimator}
Let's start with a 3D intensity source field $I_s(\theta, D_M)$, which is the unlensed field of the matter present at a transverse comoving distance $D_M$ away from the observer. 

Due to the matter present between the source and observer, the signal experiences weak lensing and in this case the signal can be written as
\beqa
I_o(\theta,D_M) = I_s(\theta + \delta\theta, D_M )
\eeqa

where $I_o$ is the observed lensed field. This equation says that the field that we observe at coordinates $(\theta,D_M)$ is actually coming from $(\theta + \delta\theta, D_M)$, where $\delta\theta = \bigtriangledown\psi$ and $\psi$ is the \textit{projected potential}. 

Taylor expansion of RHS of previous equation gives
\beqa
I_o(\theta,D_M) = I_s(\theta,D_M) + \delta\theta \cdot \bigtriangledown_\theta I_s(\theta,D_M) + \ldots
\eeqa
%where $\il(\theta,D_M)$ is the lensed, $\iul(\theta,D_M)$ the unlensed field. 
The Fourier transform of this expression is
\begin{eqnarray}
\int d^2 \theta \; e^{-i \l \cdot \theta} \; I_o(\theta,D_M) &=& \int d^2 \theta I_s(\theta,D_M) e^{-i \l \cdot \theta} + \int d^2 \theta \; e^{-i\l \cdot \theta} \delta\theta \cdot \bigtriangledown_\theta I_s(\theta,D_M)\nonumber\\
\il_o(\l,D_M) &=& \il_s(\l,D_M) + \underbrace{\int d^2 \theta \; e^{-i\l \cdot \theta} \; \delta\theta \cdot \bigtriangledown_\theta I_s(\theta,D_M)}_{A}
\end{eqnarray}
Using $\dtheta(\vec{D_M})=\bigtriangledown \psi(\vec{D_M})$
\begin{eqnarray}
A &= & \int d^2 \theta \; e^{-i\l \cdot \theta} \; \bigtriangledown_\theta \psi \cdot \bigtriangledown_\theta I_s(\theta,D_M) \\
&= & \int d^2 \theta \; e^{-i\l \cdot \theta} \; \frac{1}{(2\pi)^4}  \bigtriangledown_\theta \int d^2 l'  e^{i\l' \cdot \theta} \tilde{\psi}(\l', D_M) \cdot \bigtriangledown_\theta \int d^2 l''  e^{i\l'' \cdot \theta} \il_s(\l'',D_M)  \\
&= & \int d^2 \theta \; e^{-i\l \cdot \theta} \; \frac{1}{(2\pi)^4}  \int d^2 l' \int d^2 l''  \tilde{\psi}(\l', D_M) \il_s(\l'',D_M) \bigtriangledown_\theta e^{i\l' \cdot \theta}  \cdot  \bigtriangledown_\theta  e^{i\l'' \cdot \theta}   \\
&= & - \int d^2 \theta \; e^{-i\l \cdot \theta} \;  \frac{1}{(2\pi)^4}  \int d^2 l' \int d^2 l''  \tilde{\psi}(\l', D_M) \il_s(\l'',D_M) e^{i(\l' + \l'') \cdot \theta} \l' \cdot \l''\\
&= & -\int d^2 \theta \; \frac{1}{(2\pi)^4}  \int d^2 l' \int d^2 l''  e^{i(-\l + \l' + \l'') \cdot \theta} \l' \cdot \l'' \; \tilde{\psi}(\l', D_M) \il_s(\l'',D_M)
\end{eqnarray}
Integrating over $ \theta $
\begin{eqnarray}
A &= &- \int d^2 l' \; \frac{1}{(2\pi)^2}  \int d^2 l'' \delta(-\l + \l' + \l'') \l' \cdot \l'' \; \tilde{\psi}(\l', D_M) \il_s(\l'',D_M)
\end{eqnarray}
Integrating over $ \l' $
\begin{eqnarray}
A &= &- \frac{1}{(2\pi)^2} \int d^2 l'' \;  \l'' \cdot (\l - \l'') \; \psi(\l - \l'', D_M) I_s(\l'',D_M)
\end{eqnarray}
Therefore, we get
\begin{eqnarray}
\il_o(\l,k) &=& \il_s(\l, k) - \int {d^2l'} \frac{1}{(2\pi)^2}\il_s(\l',k)\psi(\l-\l')(\l-\l')\cdot
\l'
\end{eqnarray}
Computing the correlation function (and dropping $ k $)
\beq
\beqal
\langle \tilde I_o (\l) \tilde{I_o}^*(\textbf{m}) \rangle _{\textbf{m} \neq \textbf{l}} &= - \int {d^2 l'}  \frac{1}{(2\pi)^2}\psi(\l-\l')(\l-\l')\cdot \l'  \langle \il_s(\l^\prime) \il_s^*(\textbf{m})  \rangle 
\\
& -  \int {d^2 l'}  \frac{1}{(2\pi)^2}\psi^*(\textbf{m}-\l')(\textbf{m}-\l')\cdot \l'  \langle \il_s^*(\l^\prime) \il_s(\textbf{m}) \rangle
 \\
 &= -  \psi(\l-\textbf{m})(\l-\textbf{m})\cdot \textbf{m}  C_m 
 \\
&  -   \psi^*(\textbf{m}-\l)(\textbf{m}-\l)\cdot \l  C_l
\eeqal
\eeq

Now, assuming that $ \textbf{m} = \l- \L $ (and restoring $ k $)
\begin{eqnarray}
\langle \tilde I (\l,k) \tilde{I}^*(\l - \L, k) \rangle _{\textbf{L}  = \l - \textbf{m}} &=& -  \psi(\L, k)\L\cdot (\textbf{l} - \textbf{L})  C_{l - L, k} +   \psi^*(-\L, k)\L\cdot \l  C_{l, k} \\
&=& -  \psi(\L, k)\L \cdot \left[(\textbf{l} - \textbf{L})  C_{l - L, k} +  \cdot \l  C_{l, k}\right]
\end{eqnarray}
%Even though the field is not homogeneous, it is still assumed to be isotropic. Therefore, LHS = $ \tilde I (l) \tilde{I}^*(\l - \L) $. \textbf{make sure if it's correct}
%
%On RHS, the term multiplying $ \phi $ is just a number. 
We simply invert the expression to solve for $ \phi $ and see that (for a particular $ \l $ and $ \textbf{m} $ such that $ \l  - \textbf{m} = \L$) \textbf{?}
\beqa
\psi(\L) \propto I(\l) I(\l - \L)
\eeqa
Paragraph below equation 9.15 on Page 204 of "Gravitational Lensing" by Dodelson, says that	

The above equation is not the best way to infer the potential. First, it uses only a single small-scale model $ \l $; obviously, it makes sense to combine information from many small-scale modes, in each case taking the product of two temperatures with arguments seperated by $ \L $. Second, the estimator in the previous equation will indeed retrieve the true value of $ \phi $ on average but will be very noisy. The quadratic terms can be weighted to reduce the noise while retaining the attractive features that the expected value of the estimator is equal to the true potential.

%\subsection{Deriving the estimator}
%We therefore start with a quadratic estimator $\Phi(\L)$ for $\phi(\L)$,
%i.e. of the form
%\beq
%\Phi (\L) = \int \d2l \int \dko \int \dkt F(\l,k_1,k_2,\L)
%I(\l,k_1) I(\L-\l,k_2)
%\label{eq:quadest}
%\eeq
%(notice that $F(\l,k_1,k_2,\L)=F(\l,k_2,k_1,\L)$). Because $\delta
%\Phi(\L)=\delta \Phi^*(-\L)$ it can also be shown that
%\beq
%F(\l,k_1,k_2,\L) = F^*(-\l,-k_1,-k_2,-\L)
%\eeq
%We want to find $F$ such that it minimizes the variance of $\Phi(\L)$
%under the condition that its ensemble average recovers the lensing field,
%$\langle \Phi(\L)\rangle_\iul = \phi(\L)$. This becomes (to first order in $\phi$)
%
%\beq
%\beqal
%\langle \Phi(\L)\rangle_\iul &=&  \int \d2l \int \dko \int \dkt  F(\l,k_1,k_2,\L)  (2\pi)\delta(k_1+k_2) \\
%& & \times \left[\pul_{l,k} \phi(\L)\L\cdot\l + \pul_{L-l,k}  \phi(\L)\L\cdot(\L-\l)\right] \, ,
%\eeqal
%\eeq
%
%where e.g. $\pul_{l,k}$ is the power in a mode with angular component $l$ and radial component $k$.
%With the requirement that $\langle \Phi(\L)\rangle_\iul=\phi(\L)$ this
%leads to the normalization condition
%\beq
%\int \d2l \int \dko \int \dkt F(\l,k_1,k_2,\L) (2\pi) \delta^D(k_1+k_2)
%\left[\pul_{l,k_1}\L\cdot\l + \pul_{L-l,k_2} \L\cdot (\L-\l)\right] = 1
%\label{eq:normcond}
%\eeq
%
%The condition of minimization of the variance gives 
%\beq
%\beqal
%\langle ||\Phi(\L)||^2\rangle_\il &= \int \d2l \int \dko \int \dkt (2\pi)^2 \delta(0) F(\l,k_1,k_2,\L)
%F^*(\l',k_1',k_2',\L') \ptot_{l,k_1}\ptot_{L-l,k_2} \\
%&  + \int \d2l \int \dko \int \dkt (2\pi)^2 \delta(0) F(\l,k_1,k_2,\L)
%F^*(\L-\l',k_2',k_1',\L') \ptot_{l,k_1}\ptot_{L-l,k_2}
%\eeqal
%\eeq
%
%but from \ref{eq:quadest} we see with the substitution $\L-\l
%\rightarrow \l$ that $F^*(\L-\l,k_2,k_1,\L) = F^*(\l,k_1,k_2,\L)$
%hence
%\beq
%\langle ||\Phi(\L)||^2\rangle = 2 (2\pi)^2 \delta(0) \int \d2l \int \dko \int
%\dkt F(\l,k_1,k_2,\L) F^*(\l,k_1,k_2,\L) \ptot_{l,k_1}\ptot_{l,k_2} \, .  \label{eq:phisqavg}
%\eeq
%
%Both real and imaginary part of $||F||^2=F_R^2+F_I^2$ contribute
%to this variance, however the condition for the
%minimization will only pick out the real part. (\textbf{Does this really matter?}) The solution is found
%by minimizing the function
%\beq
%\langle ||\Phi(\L)||^2\rangle- A_R \times (\, {\rm Equation} \, \ref{eq:normcond})
%\eeq
%with respect to $F$, where $A_R$ is a Lagrangian multiplier. In steps,
%\beq
%\frac{\partial (\, {\rm Eq.}\, \ref{eq:normcond})}{\partial F(\l,k_1,k_2,\L)} = A_R \d2l \dko \dkt
%(2\pi) \delta^D(k_1+k_2) [\pul_{l,k_1} \L\cdot \l + \pul_{L-l,k_2} \L\cdot (\L-\l)]
%\eeq
%and
%\beq
%\frac{\partial \langle ||\Phi(\L)||^2 \rangle}{\partial F(\l,k_1,k_2,\L)} =
%2 (2\pi)^2 \delta(0) \d2l \dko \dkt 2 F_R(\l,k_1,k_2,\L)
%\ptot_{l,k_1}\ptot_{L-l,k_2}
%\eeq
%so
%\beq
%\beqal
%4 (2\pi)^2 \delta(0)  F_R(\l,k_1,k_2,\L) = A_R (2\pi) \delta^D(k_1+k_2) \frac{[\pul_{l,k_1} \L\cdot \l +
%    \pul_{L-l,k_2} \L\cdot (\L-\l)]}{\ptot_{l,k_1}\ptot_{L-l,k_2}}
%\label{eq:expforf}
%\eeqal
%\eeq
%
%Therefore,
%
%\beq
%\beqal
%F_R(\l,k_1,k_2,\L) = \frac{1}{4 (2\pi) \delta(0) } A_R \delta^D(k_1+k_2) \frac{[\pul_{l,k_1} \L\cdot \l +
%	\pul_{L-l,k_2} \L\cdot (\L-\l)]}{\ptot_{l,k_1}\ptot_{L-l,k_2}}
%\label{eq:expforf}
%\eeqal
%\eeq
%
%
%and by inserting this into the normalization condition
%\ref{eq:normcond} we get that
%\beq
%\beqal
%1 &= \int \d2l \int \dko \int \dkt \frac{1}{4 \delta(0) } A_R \delta^D(k_1+k_2) \frac{[\pul_{l,k_1} \L\cdot \l +
%	\pul_{L-l,k_2} \L\cdot (\L-\l)]^2}{\ptot_{l,k_1}\ptot_{L-l,k_2}} \delta^D(k_1+k_2) 
%\\
%& = \int \d2l \int \dko \frac{\delta(0)}{4 \delta(0) } A_R \frac{[\pul_{l,k_1} \L\cdot \l +
%	\pul_{L-l,k_2 = -k_1} \L\cdot (\L-\l)]^2}{\ptot_{l,k_1}\ptot_{L-l,k_2 = -k_1}} 
%%\\
%%& = \int \frac{d^2l}{(2\pi)^2} \int dk_1 \frac{\delta(0)}{4(2\pi) \delta(0)} A_R [\frac{P_{l,k_1} \textbf{L}\cdot \textbf{l}}{\tilde{P}^{tot}_{l,k_1} \tilde{P}^{\tot}_{L-l, -k_1}} + \frac{P_{l,k_1} \textbf{L}\cdot \textbf{L} - \textbf{l}}{\tilde{P}^{tot}_{l,k_1} \tilde{P}^{\tot}_{L-l, -k_1}}]
%\eeqal
%\eeq
%
%Assuming that I can just cancel the factors of $ \delta(0) $, 
%\beq
%\beqal
%1 = \int \d2l \int dk_1 \frac{1}{4\times2\pi} A_R \frac{[\pul_{l,k_1} \L\cdot \l +
%	\pul_{L-l,-k_1} \L\cdot (\L-\l)]^2}{\ptot_{l,k_1}\ptot_{L-l,-k_1}} 
%\eeqal
%\eeq
%
%Discretizing the integal over $ dk_1 $. Assumming that we have dicretized the space in the radial direction in the blocks of length $ D_C $ such that $ k = \frac{2\pi}{D_C} $.
%
%Therefore, we can write 
%\beqa
%\int \frac{dk}{2 \pi} = \frac{1}{D_C} \sum_k
%\eeqa
%
%Therefore, we get
%\beqa
%1 = \int \d2l \frac{1}{D_C} \sum_{k_1} \frac{1}{4 (2\pi)} A_R \frac{[\pul_{l,k_1} \L\cdot \l +
%	\pul_{L-l,-k_1} \L\cdot (\L-\l)]^2}{\ptot_{l,k_1}\ptot_{L-l,-k_1}} 
%\eeqa
%
%which gives, 
%\beqa
%A_R = \frac{4 (2\pi)D_C}{\int \frac{d^2 l}{(2\pi)^2} \sum_{k_1} \frac{[\pul_{l,k_1} \L\cdot \l +
%		\pul_{L-l,-k_1} \L\cdot (\L-\l)]^2}{\ptot_{l,k_1}\ptot_{L-l,-k_1}} }  
%\eeqa
%
%%\textbf{I don't understand how the authors (Zahn and Zaldarriaga) got the following expression}
%%
%%\beq
%%A_R = \frac{1}{\sum_k \int \d2l \frac{[\pul_{l,k} \L\cdot \l +
%%    \pul_{L-l,k} \L\cdot (\L-\l)]^2}{\ptot_{l,k_1}\ptot_{L-l,k_2}}} \, .
%%\eeq
%
%Substituting this expression in Equation \ref{eq:expforf}, we get
%
%\beq
%\beqal
%F_R(\l,k_1,k_2,\L) &= \frac{A_R }{4 (2\pi) \delta(0) }\delta^D(k_1+k_2) \frac{[\pul_{l,k_1} \L\cdot \l +
%	\pul_{L-l,k_2} \L\cdot (\L-\l)]}{\ptot_{l,k_1}\ptot_{L-l,k_2}}
%\\
%&=  \frac{D_C / \delta(0)}{\int \frac{d^2 l}{(2\pi)^2} \sum_{k_1} \frac{[\pul_{l,k_1} \L\cdot \l +
%		\pul_{L-l,-k_1} \L\cdot (\L-\l)]^2}{\ptot_{l,k_1}\ptot_{L-l,-k_1}} }   \delta^D(k_1+k_2) \frac{[\pul_{l,k_1} \L\cdot \l +
%	\pul_{L-l,k_2} \L\cdot (\L-\l)]}{\ptot_{l,k_1}\ptot_{L-l,k_2}}
%\eeqal
%\eeq
%
%Now, substituting this expression in Equation \ref{eq:phisqavg}. Taking $ FF^* \equiv F^2 $
%
%\beq
%\beqal
%\langle ||\Phi(\L)||^2\rangle &= 2 (2\pi)^2 \delta(0) \int \d2l \int \dko \int
%\dkt F(\l,k_1,k_2,\L) F^*(\l,k_1,k_2,\L) \ptot_{l,k_1}\ptot_{l,k_2} 
%\\
%&= 2 (2\pi)^2 \delta(0) \int \d2l \int \dko \int
%\dkt F^2(\l,k_1,k_2,\L)\ptot_{l,k_1}\ptot_{l,k_2} 
%\\
%&= 2 (2\pi)^2 \delta(0)  \int \d2l \int \dko \int \dkt \left( \frac{D_C / \delta(0)}{\int \frac{d^2 l}{(2\pi)^2} \sum_{k_1} \frac{[\pul_{l,k_1} \L\cdot \l +
%		\pul_{L-l,-k_1} \L\cdot (\L-\l)]^2}{\ptot_{l,k_1}\ptot_{L-l,-k_1}} }   \right.
%\\
%& \left.	\delta^D(k_1+k_2) \frac{[\pul_{l,k_1} \L\cdot \l +
%	\pul_{L-l,k_2} \L\cdot (\L-\l)]}{\ptot_{l,k_1}\ptot_{L-l,k_2}} \right)^2 \ptot_{l,k_1}\ptot_{l,k_2} 
%\eeqal
%\eeq
%
%Again, cancelling $ \delta(0) $,
%
%\beq
%\beqal
%\langle ||\Phi(\L)||^2\rangle 
%&=  \frac{2 (2\pi)^2}{\delta(0)}   \int \d2l \int \dko \int \dkt \left( \frac{D_C}{\int \frac{d^2 l}{(2\pi)^2} \sum_{k_1} \frac{[\pul_{l,k_1} \L\cdot \l +
%		\pul_{L-l,-k_1} \L\cdot (\L-\l)]^2}{\ptot_{l,k_1}\ptot_{L-l,-k_1}} }   \right.
%\\
%& \left.	\delta^D(k_1+k_2) \frac{[\pul_{l,k_1} \L\cdot \l +
%	\pul_{L-l,k_2} \L\cdot (\L-\l)]}{\ptot_{l,k_1}\ptot_{L-l,k_2}} \right)^2 \ptot_{l,k_1}\ptot_{l,k_2} 
%\\
%&=  \frac{2 (2\pi)^2}{\delta(0)} \left( \frac{D_C}{\int \frac{d^2 l}{(2\pi)^2} \sum_{k_1} \frac{[\pul_{l,k_1} \L\cdot \l +
%		\pul_{L-l,-k_1} \L\cdot (\L-\l)]^2}{\ptot_{l,k_1}\ptot_{L-l,-k_1}} } \right)^2 \int \d2l \int \dko \int \dkt 
%\\
%& \left(	\delta^D(k_1+k_2) \frac{[\pul_{l,k_1} \L\cdot \l +
%	\pul_{L-l,k_2} \L\cdot (\L-\l)]}{\ptot_{l,k_1}\ptot_{L-l,k_2}} \right)^2 \ptot_{l,k_1}\ptot_{l,k_2} 
%\eeqal
%\eeq
%
%integrating over $ k_2 $
%
%\beq
%\beqal
%\langle ||\Phi(\L)||^2\rangle 
%&=  \frac{2 (2\pi)^2}{\delta(0)} \left( \frac{D_C}{\int \frac{d^2 l}{(2\pi)^2} \sum_{k_1} \frac{[\pul_{l,k_1} \L\cdot \l +
%		\pul_{L-l,-k_1} \L\cdot (\L-\l)]^2}{\ptot_{l,k_1}\ptot_{L-l,-k_1}} } \right)^2 \int \d2l \int \dko \frac{1}{2\pi} 
%\\
%&  \delta(0) \left( \frac{[\pul_{l,k_1} \L\cdot \l +
%	\pul_{L-l,k_2 = -k_1} \L\cdot (\L-\l)]}{\ptot_{l,k_1}\ptot_{L-l,k_2 = -k_1}} \right)^2 \ptot_{l,k_1}\ptot_{l,k_2 = -k_1} 
%\eeqal
%\eeq
%
%cancelling $ \delta(0) $ and $ 2\pi $
%\beq
%\beqal
%\langle ||\Phi(\L)||^2\rangle 
%&=  2 (2\pi) \left( \frac{D_C}{\int \frac{d^2 l}{(2\pi)^2} \sum_{k_1} \frac{[\pul_{l,k_1} \L\cdot \l +
%		\pul_{L-l,-k_1} \L\cdot (\L-\l)]^2}{\ptot_{l,k_1}\ptot_{L-l,-k_1}} } \right)^2 \int \d2l \int \dko  
%\\
%&  \left( \frac{[\pul_{l,k_1} \L\cdot \l +
%	\pul_{L-l,-k_1} \L\cdot (\L-\l)]}{\ptot_{l,k_1}\ptot_{L-l, -k_1}} \right)^2 \ptot_{l,k_1}\ptot_{l,-k_1} 
%\eeqal
%\eeq
%
%
%Discretizing the integral over $ k_1 $. 
%
%\beq
%\beqal
%\langle ||\Phi(\L)||^2\rangle 
%&=  2 (2\pi) \left( \frac{D_C}{\int \frac{d^2 l}{(2\pi)^2} \sum_{k_1} \frac{[\pul_{l,k_1} \L\cdot \l +
%		\pul_{L-l,-k_1} \L\cdot (\L-\l)]^2}{\ptot_{l,k_1}\ptot_{L-l,-k_1}} } \right)^2 \int \d2l   
%\\
%& \frac{1}{D_C} \sum_{k_1} \left( \frac{[\pul_{l,k_1} \L\cdot \l +
%	\pul_{L-l,-k_1} \L\cdot (\L-\l)]}{\ptot_{l,k_1}\ptot_{L-l,-k_1}} \right)^2 \ptot_{l,k_1}\ptot_{l,-k_1} 
%\\
%&=  2 (2\pi) \left( \frac{D_C}{\int \frac{d^2 l}{(2\pi)^2} \sum_{k_1} \frac{[\pul_{l,k_1} \L\cdot \l +
%		\pul_{L-l,-k_1} \L\cdot (\L-\l)]^2}{\ptot_{l,k_1}\ptot_{L-l,-k_1}} } \right)^2    
%\\
%&  \left( \int \d2l \frac{1}{D_C} \sum_{k_1} \frac{[\pul_{l,k_1} \L\cdot \l +
%	\pul_{L-l,-k_1} \L\cdot (\L-\l)] ^2}{\ptot_{l,k_1}\ptot_{l,-k_1} }  \right)
%\\
%&= 2 (2\pi) D_C \left( \frac{1}{\int \frac{d^2 l}{(2\pi)^2} \sum_{k_1} \frac{[\pul_{l,k_1} \L\cdot \l +
%		\pul_{L-l,-k_1} \L\cdot (\L-\l)]^2}{\ptot_{l,k_1}\ptot_{L-l,-k_1}} } \right)^2    
%\\
%&  \left( \int \d2l \sum_{k_1} \frac{[\pul_{l,k_1} \L\cdot \l +
%	\pul_{L-l, -k_1} \L\cdot (\L-\l)] ^2}{\ptot_{l,k_1}\ptot_{l,-k_1} }  \right)
%\eeqal
%\eeq
%
%Therefore, we finally have
%\beq
%\beqal
%\langle ||\Phi(\L)||^2\rangle 
%& =   \frac{2 (2\pi) D_C}{\int \frac{d^2 l}{(2\pi)^2} \sum_{k_1} \frac{[\pul_{l,k_1} \L\cdot \l +
%		\pul_{L-l,-k_1} \L\cdot (\L-\l)]^2}{\ptot_{l,k_1}\ptot_{L-l,-k_1}} }
%\\
%& = \frac{A_R}{2} 
%\eeqal
%\eeq
%
%With the definition of noise power spectrum
%\beq
%\beqal
%\langle \Phi(\textbf{L})  \Phi^*(\textbf{L}')\rangle = (2\pi)^2 \delta(\textbf{L} - \textbf{L}') N_L^\Phi
%\\
%\mathrm{OR}
%\\
%\langle || \Phi(\textbf{L}) ||^2 \rangle = (2\pi)^2 \delta(0) N_L^\Phi
%\eeqal
%\eeq
%
%It follows that 
%\beq
%\beqal
%N_L^\Phi = \frac{2 D_C}{2\pi \delta(0)} \frac{1}{{\int \frac{d^2 l}{(2\pi)^2} \sum_{k_1} \frac{[\pul_{l,k_1} \L\cdot \l +
%			\pul_{L-l,-k_1} \L\cdot (\L-\l)]^2}{\ptot_{l,k_1}\ptot_{L-l,-k_1}} }}
%\eeqal
%\eeq


\subsection{Deriving the estimator}
We therefore start with a quadratic estimator $\Phi(\L)$ for $\phi(\L)$,
i.e. of the form
\beq
\Phi (\L) = \int \d2l \int \dko \int \dkt F(\l,k_1,k_2,\L)
I(\l,k_1) I(\L-\l,k_2)
\label{eq:quadest}
\eeq
(notice that $F(\l,k_1,k_2,\L)=F(\l,k_2,k_1,\L)$). Because $\delta
\Phi(\L)=\delta \Phi^*(-\L)$ it can also be shown that
\beq
F(\l,k_1,k_2,\L) = F^*(-\l,-k_1,-k_2,-\L)
\eeq
We want to find $F$ such that it minimizes the variance of $\Phi(\L)$
under the condition that its ensemble average recovers the lensing field,
$\langle \Phi(\L)\rangle_\iul = \phi(\L)$. This becomes (to first order in $\phi$)

\beq
\beqal
\langle \Phi(\L)\rangle_\iul &=&  \int \d2l \int \dko \int \dkt  F(\l,k_1,k_2,\L)  (2\pi)\delta(k_1+k_2) \\
& & \times \left[\pul_{l,k} \phi(\L)\L\cdot\l + \pul_{L-l,k}  \phi(\L)\L\cdot(\L-\l)\right] \, ,
\eeqal
\eeq

where e.g. $\pul_{l,k}$ is the power in a mode with angular component $l$ and radial component $k$.
With the requirement that $\langle \Phi(\L)\rangle_\iul=\phi(\L)$ this
leads to the normalization condition
\beq
\int \d2l \int \dko \int \dkt F(\l,k_1,k_2,\L) (2\pi) \delta^D(k_1+k_2)
\left[\pul_{l,k_1}\L\cdot\l + \pul_{L-l,k_2} \L\cdot (\L-\l)\right] = 1
\label{eq:normcond}
\eeq

The condition of minimization of the variance gives 
\beq
\beqal
\langle ||\Phi(\L)||^2\rangle_\il &= \int \d2l \int \dko \int \dkt (2\pi)^2 \delta(0) F(\l,k_1,k_2,\L)
F^*(\l',k_1',k_2',\L') \ptot_{l,k_1}\ptot_{L-l,k_2} \\
&  + \int \d2l \int \dko \int \dkt (2\pi)^2 \delta(0) F(\l,k_1,k_2,\L)
F^*(\L-\l',k_2',k_1',\L') \ptot_{l,k_1}\ptot_{L-l,k_2}
\eeqal
\eeq

but from \ref{eq:quadest} we see with the substitution $\L-\l
\rightarrow \l$ that $F^*(\L-\l,k_2,k_1,\L) = F^*(\l,k_1,k_2,\L)$
hence
\beq
\langle ||\Phi(\L)||^2\rangle = 2 (2\pi)^2 \delta(0) \int \d2l \int \dko \int
\dkt F(\l,k_1,k_2,\L) F^*(\l,k_1,k_2,\L) \ptot_{l,k_1}\ptot_{l,k_2} \, .  \label{eq:phisqavg}
\eeq

Both real and imaginary part of $||F||^2=F_R^2+F_I^2$ contribute
to this variance, however the condition for the
minimization will only pick out the real part. (\textbf{Does this really matter?}) The solution is found
by minimizing the function
\beq
\langle ||\Phi(\L)||^2\rangle- A_R \times (\, {\rm Equation} \, \ref{eq:normcond})
\eeq
with respect to $F$, where $A_R$ is a Lagrangian multiplier. In steps,
\beq
\frac{\partial (\, {\rm Eq.}\, \ref{eq:normcond})}{\partial F(\l,k_1,k_2,\L)} = A_R \d2l \dko \dkt
(2\pi) \delta^D(k_1+k_2) [\pul_{l,k_1} \L\cdot \l + \pul_{L-l,k_2} \L\cdot (\L-\l)]
\eeq
and
\beq
\frac{\partial \langle ||\Phi(\L)||^2 \rangle}{\partial F(\l,k_1,k_2,\L)} =
2 (2\pi)^2 \delta(0) \d2l \dko \dkt 2 F_R(\l,k_1,k_2,\L)
\ptot_{l,k_1}\ptot_{L-l,k_2}
\eeq
so
\beq
\beqal
4 (2\pi)^2 \delta(0)  F_R(\l,k_1,k_2,\L) = A_R (2\pi) \delta^D(k_1+k_2) \frac{[\pul_{l,k_1} \L\cdot \l +
	\pul_{L-l,k_2} \L\cdot (\L-\l)]}{\ptot_{l,k_1}\ptot_{L-l,k_2}}
\label{eq:expforf}
\eeqal
\eeq

Therefore,

\beq
\beqal
F_R(\l,k_1,k_2,\L) = \frac{1}{4 (2\pi) \delta(0) } A_R \delta^D(k_1+k_2) \frac{[\pul_{l,k_1} \L\cdot \l +
	\pul_{L-l,k_2} \L\cdot (\L-\l)]}{\ptot_{l,k_1}\ptot_{L-l,k_2}}
\label{eq:expforf}
\eeqal
\eeq


and by inserting this into the normalization condition
\ref{eq:normcond} we get that
\beq
\beqal
1 &= \int \d2l \int \dko \int \dkt \frac{1}{4 \delta(0) } A_R \delta^D(k_1+k_2) \frac{[\pul_{l,k_1} \L\cdot \l +
	\pul_{L-l,k_2} \L\cdot (\L-\l)]^2}{\ptot_{l,k_1}\ptot_{L-l,k_2}} \delta^D(k_1+k_2) 
\\
& = \int \d2l \int \dko \frac{\delta(0)}{4 \delta(0) } A_R \frac{[\pul_{l,k_1} \L\cdot \l +
	\pul_{L-l,k_2 = -k_1} \L\cdot (\L-\l)]^2}{\ptot_{l,k_1}\ptot_{L-l,k_2 = -k_1}} 
%\\
%& = \int \frac{d^2l}{(2\pi)^2} \int dk_1 \frac{\delta(0)}{4(2\pi) \delta(0)} A_R [\frac{P_{l,k_1} \textbf{L}\cdot \textbf{l}}{\tilde{P}^{tot}_{l,k_1} \tilde{P}^{\tot}_{L-l, -k_1}} + \frac{P_{l,k_1} \textbf{L}\cdot \textbf{L} - \textbf{l}}{\tilde{P}^{tot}_{l,k_1} \tilde{P}^{\tot}_{L-l, -k_1}}]
\eeqal
\eeq

Assuming that I can just cancel the factors of $ \delta(0) $, 
\beq
\beqal
1 = \int \d2l \int dk_1 \frac{1}{4\times2\pi} A_R \frac{[\pul_{l,k_1} \L\cdot \l +
	\pul_{L-l,-k_1} \L\cdot (\L-\l)]^2}{\ptot_{l,k_1}\ptot_{L-l,-k_1}} 
\eeqal
\eeq

Discretizing the integal over $ dk_1 $. Assumming that we have dicretized the space in the radial direction in the blocks of length $ D_C $ such that $ k = \frac{2\pi}{D_C} $.

Therefore, we can write 
\beqa
\int \frac{dk}{2 \pi} = \frac{1}{D_C} \sum_k
\eeqa

Therefore, we get
\beqa
1 = \int \d2l \frac{1}{D_C} \sum_{k_1} \frac{1}{4} A_R \frac{[\pul_{l,k_1} \L\cdot \l +
	\pul_{L-l,-k_1} \L\cdot (\L-\l)]^2}{\ptot_{l,k_1}\ptot_{L-l,-k_1}} 
\eeqa

which gives, 
\beqa
A_R = \frac{4 D_C}{\int \frac{d^2 l}{(2\pi)^2} \sum_{k_1} \frac{[\pul_{l,k_1} \L\cdot \l +
		\pul_{L-l,-k_1} \L\cdot (\L-\l)]^2}{\ptot_{l,k_1}\ptot_{L-l,-k_1}} }  
\eeqa

%\textbf{I don't understand how the authors (Zahn and Zaldarriaga) got the following expression}
%
%\beq
%A_R = \frac{1}{\sum_k \int \d2l \frac{[\pul_{l,k} \L\cdot \l +
%    \pul_{L-l,k} \L\cdot (\L-\l)]^2}{\ptot_{l,k_1}\ptot_{L-l,k_2}}} \, .
%\eeq

Substituting this expression in Equation \ref{eq:expforf}, we get

\beq
\beqal
F_R(\l,k_1,k_2,\L) &= \frac{A_R }{4 (2\pi) \delta(0) }\delta^D(k_1+k_2) \frac{[\pul_{l,k_1} \L\cdot \l +
	\pul_{L-l,k_2} \L\cdot (\L-\l)]}{\ptot_{l,k_1}\ptot_{L-l,k_2}}
\\
&=  \frac{D_C / 2\pi \delta(0)}{\int \frac{d^2 l}{(2\pi)^2} \sum_{k_1} \frac{[\pul_{l,k_1} \L\cdot \l +
		\pul_{L-l,-k_1} \L\cdot (\L-\l)]^2}{\ptot_{l,k_1}\ptot_{L-l,-k_1}} }   \delta^D(k_1+k_2) \frac{[\pul_{l,k_1} \L\cdot \l +
	\pul_{L-l,k_2} \L\cdot (\L-\l)]}{\ptot_{l,k_1}\ptot_{L-l,k_2}}\\
&=  \frac{1}{\int \frac{d^2 l}{(2\pi)^2} \sum_{k_1} \frac{[\pul_{l,k_1} \L\cdot \l +
		\pul_{L-l,-k_1} \L\cdot (\L-\l)]^2}{\ptot_{l,k_1}\ptot_{L-l,-k_1}} }   \delta^D(k_1+k_2) \frac{[\pul_{l,k_1} \L\cdot \l +
	\pul_{L-l,k_2} \L\cdot (\L-\l)]}{\ptot_{l,k_1}\ptot_{L-l,k_2}}
\eeqal
\eeq

Now, substituting this expression in Equation \ref{eq:phisqavg}. Taking $ FF^* \equiv F^2 $

\beq
\beqal
\langle ||\Phi(\L)||^2\rangle &= 2 (2\pi)^2 \delta(0) \int \d2l \int \dko \int
\dkt F(\l,k_1,k_2,\L) F^*(\l,k_1,k_2,\L) \ptot_{l,k_1}\ptot_{l,k_2} 
\\
&= 2 (2\pi)^2 \delta(0) \int \d2l \int \dko \int
\dkt F^2(\l,k_1,k_2,\L)\ptot_{l,k_1}\ptot_{l,k_2} 
\\
&= 2 (2\pi)^2 \delta(0)  \int \d2l \int \dko \int \dkt \left( \frac{1}{\int \frac{d^2 l}{(2\pi)^2} \sum_{k_1} \frac{[\pul_{l,k_1} \L\cdot \l +
		\pul_{L-l,-k_1} \L\cdot (\L-\l)]^2}{\ptot_{l,k_1}\ptot_{L-l,-k_1}} }   \right.
\\
& \left.	\delta^D(k_1+k_2) \frac{[\pul_{l,k_1} \L\cdot \l +
	\pul_{L-l,k_2} \L\cdot (\L-\l)]}{\ptot_{l,k_1}\ptot_{L-l,k_2}} \right)^2 \ptot_{l,k_1}\ptot_{l,k_2} 
\\
&=  2 (2\pi)^2\delta(0) \int \d2l \int \dko \int \dkt \left( \frac{1}{\int \frac{d^2 l}{(2\pi)^2} \sum_{k_1} \frac{[\pul_{l,k_1} \L\cdot \l +
		\pul_{L-l,-k_1} \L\cdot (\L-\l)]^2}{\ptot_{l,k_1}\ptot_{L-l,-k_1}} } \right)^2
\\
& \left(	\delta^D(k_1+k_2) \frac{[\pul_{l,k_1} \L\cdot \l +
	\pul_{L-l,k_2} \L\cdot (\L-\l)]}{\ptot_{l,k_1}\ptot_{L-l,k_2}} \right)^2 \ptot_{l,k_1}\ptot_{l,k_2} 
\\
&=  2 (2\pi)^2\delta(0) \int \d2l \frac{\sum_{k_1}}{D_C} \frac{\sum_{k_2}}{D_C} \left( \frac{1}{\int \frac{d^2 l}{(2\pi)^2} \sum_{k_1} \frac{[\pul_{l,k_1} \L\cdot \l +
		\pul_{L-l,-k_1} \L\cdot (\L-\l)]^2}{\ptot_{l,k_1}\ptot_{L-l,-k_1}} } \right)^2
\\
& \left(	\delta^D(k_1+k_2) \frac{[\pul_{l,k_1} \L\cdot \l +
	\pul_{L-l,k_2} \L\cdot (\L-\l)]}{\ptot_{l,k_1}\ptot_{L-l,k_2}} \right)^2 \ptot_{l,k_1}\ptot_{l,k_2} 
\\
	&=  2 (2\pi)^2\delta(0) \frac{1}{\int \frac{d^2 l}{(2\pi)^2} \sum_{k_1} \frac{[\pul_{l,k_1} \L\cdot \l +
		\pul_{L-l,-k_1} \L\cdot (\L-\l)]^2}{\ptot_{l,k_1}\ptot_{L-l,-k_1}} } 
\eeqal
\eeq
%
%
%\beq
%\beqal
%\langle ||\Phi(\L)||^2\rangle 
%&=  2 (2\pi)^2\delta(0)   \int \d2l \int \dko \int \dkt \frac{1}{\int \frac{d^2 l}{(2\pi)^2} 2 (2\pi)^2\delta(0)\sum_{k_1} \frac{[\pul_{l,k_1} \L\cdot \l +
%		\pul_{L-l,-k_1} \L\cdot (\L-\l)]^2}{\ptot_{l,k_1}\ptot_{L-l,-k_1}} }   
%\eeqal
%\eeq
%integrating over $ k_2 $
%
%\beq
%\beqal
%\langle ||\Phi(\L)||^2\rangle 
%&=  \frac{2 (2\pi)^2}{\delta(0)} \left( \frac{D_C}{\int \frac{d^2 l}{(2\pi)^2} \sum_{k_1} \frac{[\pul_{l,k_1} \L\cdot \l +
%		\pul_{L-l,-k_1} \L\cdot (\L-\l)]^2}{\ptot_{l,k_1}\ptot_{L-l,-k_1}} } \right)^2 \int \d2l \int \dko \frac{1}{2\pi} 
%\\
%&  \delta(0) \left( \frac{[\pul_{l,k_1} \L\cdot \l +
%	\pul_{L-l,k_2 = -k_1} \L\cdot (\L-\l)]}{\ptot_{l,k_1}\ptot_{L-l,k_2 = -k_1}} \right)^2 \ptot_{l,k_1}\ptot_{l,k_2 = -k_1} 
%\eeqal
%\eeq
%
%cancelling $ \delta(0) $ and $ 2\pi $
%\beq
%\beqal
%\langle ||\Phi(\L)||^2\rangle 
%&=  2 (2\pi) \left( \frac{D_C}{\int \frac{d^2 l}{(2\pi)^2} \sum_{k_1} \frac{[\pul_{l,k_1} \L\cdot \l +
%		\pul_{L-l,-k_1} \L\cdot (\L-\l)]^2}{\ptot_{l,k_1}\ptot_{L-l,-k_1}} } \right)^2 \int \d2l \int \dko  
%\\
%&  \left( \frac{[\pul_{l,k_1} \L\cdot \l +
%	\pul_{L-l,-k_1} \L\cdot (\L-\l)]}{\ptot_{l,k_1}\ptot_{L-l, -k_1}} \right)^2 \ptot_{l,k_1}\ptot_{l,-k_1} 
%\eeqal
%\eeq
%
%
%Discretizing the integral over $ k_1 $. 
%
%\beq
%\beqal
%\langle ||\Phi(\L)||^2\rangle 
%&=  2 (2\pi) \left( \frac{D_C}{\int \frac{d^2 l}{(2\pi)^2} \sum_{k_1} \frac{[\pul_{l,k_1} \L\cdot \l +
%		\pul_{L-l,-k_1} \L\cdot (\L-\l)]^2}{\ptot_{l,k_1}\ptot_{L-l,-k_1}} } \right)^2 \int \d2l   
%\\
%& \frac{1}{D_C} \sum_{k_1} \left( \frac{[\pul_{l,k_1} \L\cdot \l +
%	\pul_{L-l,-k_1} \L\cdot (\L-\l)]}{\ptot_{l,k_1}\ptot_{L-l,-k_1}} \right)^2 \ptot_{l,k_1}\ptot_{l,-k_1} 
%\\
%&=  2 (2\pi) \left( \frac{D_C}{\int \frac{d^2 l}{(2\pi)^2} \sum_{k_1} \frac{[\pul_{l,k_1} \L\cdot \l +
%		\pul_{L-l,-k_1} \L\cdot (\L-\l)]^2}{\ptot_{l,k_1}\ptot_{L-l,-k_1}} } \right)^2    
%\\
%&  \left( \int \d2l \frac{1}{D_C} \sum_{k_1} \frac{[\pul_{l,k_1} \L\cdot \l +
%	\pul_{L-l,-k_1} \L\cdot (\L-\l)] ^2}{\ptot_{l,k_1}\ptot_{l,-k_1} }  \right)
%\\
%&= 2 (2\pi) D_C \left( \frac{1}{\int \frac{d^2 l}{(2\pi)^2} \sum_{k_1} \frac{[\pul_{l,k_1} \L\cdot \l +
%		\pul_{L-l,-k_1} \L\cdot (\L-\l)]^2}{\ptot_{l,k_1}\ptot_{L-l,-k_1}} } \right)^2    
%\\
%&  \left( \int \d2l \sum_{k_1} \frac{[\pul_{l,k_1} \L\cdot \l +
%	\pul_{L-l, -k_1} \L\cdot (\L-\l)] ^2}{\ptot_{l,k_1}\ptot_{l,-k_1} }  \right)
%\eeqal
%\eeq
With the definition
\begin{eqnarray}
\langle \Phi(\textbf{L}) \Phi^*(L') \rangle  = (2\pi)^2 \delta^D(\textbf{L} - \textbf{L}') N^\Phi
\end{eqnarray}

Therefore, we finally have
\beq
\beqal
\langle ||\Phi(\L)||^2\rangle 
& =   \frac{2 (2\pi) D_C}{\int \frac{d^2 l}{(2\pi)^2} \sum_{k_1} \frac{[\pul_{l,k_1} \L\cdot \l +
		\pul_{L-l,-k_1} \L\cdot (\L-\l)]^2}{\ptot_{l,k_1}\ptot_{L-l,-k_1}} }
\\
& = (2\pi)^2 \delta(0) N^\Phi
\eeqal
\eeq

Which gives
\beq
\beqal
N_L^\Phi = \frac{1}{{\int \frac{d^2 l}{(2\pi)^2} \sum_{k_1} \frac{[\pul_{l,k_1} \L\cdot \l +
			\pul_{L-l,-k_1} \L\cdot (\L-\l)]^2}{2\ptot_{l,k_1}\ptot_{L-l,-k_1}} }}
\eeqal
\eeq
This is equation A20 in ZZ2006.


\section{Notes by Metcalf and Alkistis}
\subsection{Equation 1}
\subsubsection{Discrete and Continuous FT}
For a 3D signal, we have 
\begin{eqnarray}
	\il(\textbf{k}) = \int d^3\textbf{x} e^{-i \textbf{k} \cdot  \textbf{x}} I(\textbf{x})
\end{eqnarray}
We are probing a volume of dimensions ($ \theta D_M \times \theta D_M \times D_C $), where $ D_C $ is the length of the volume along the line of sight between $ z = z_1$ and $z = z_2 $.  We divide the it into $ N_\perp $ parts along the transverse direction and $ N_\parallel $ parts along the radial direction. We discretize $ \il(k) $ 
\begin{eqnarray}
	\il(\textbf {k}) &=& \sum_{t_1} \frac{\Theta_s D_M}{N_\perp} \sum_{t_2} \frac{\Theta_s D_M }{N_\perp} \sum_{p} \frac{D_C}{N_\parallel} e^{-i \textbf {k} \cdot ( \frac{D_C}{N_\parallel}p + \frac{\Theta_s}{N_\perp} t_1 + \frac{\Theta_s}{N_\perp} t_2)} I(\textbf {x}) 
\end{eqnarray}
Using
\begin{eqnarray}
	\Theta_s \times \Theta_s &\equiv & \Omega_s \\
	N_\perp^2 &\equiv& N_\perp, 
\end{eqnarray}
we get,
\begin{eqnarray}
	\il(\textbf{k}) =  \sum_{\textbf {x}} \frac{\Omega_s D_M^2}{N_\perp^2} \frac{ D_C }{N_\parallel} e^{-i \textbf {k} \cdot \textbf {x}} I(\textbf {x})
\end{eqnarray}
Upon normalizing this result with a factor of $ D_M^2 D_C $, we get
\begin{eqnarray}
	\tilde{I}(\textbf{k}) =  \sum_{\textbf {x}} \frac{\Omega_s}{N_\perp N_\parallel} e^{-i \textbf {k} \cdot \textbf {x}} I(\textbf {x})
\end{eqnarray}


\subsection{Equation 2}
Writing the inverse FT, we get
\begin{eqnarray}
	I(\textbf{x}) = \frac{1}{(2\pi)^3}  \int d^3 \textbf{k} e^{i \textbf{k} \cdot \textbf{x}} \il(\textbf{k})
\end{eqnarray}
%We have divided the block of space into ($ N_\perp \times N_\perp \times N_\parallel $) parts. Therefore, now the smallest lengths are equal to $\frac{dz}{N_\parallel} $ and $ \frac{\Theta_s}{N_\perp} $. Corresponding to these lengths, the $ k $ modes are $ \frac{2\pi}{dz} N_\parallel $ and $ \frac{2\pi}{\Theta_s} N_\perp  $. Using this, we get
Upon discretization, we get
\begin{eqnarray}
	I(\textbf{x}) &=& \frac{1}{(2 \pi)^3} \sum_{\textbf{k}} \frac{2\pi}{D_C} \frac{(2\pi)^2}{\Omega_s D_M^2} e^{i \textbf{k} \cdot \textbf{x}} \il(\textbf{k}) 
\end{eqnarray}
Absorbing the factor of $ D_M^2 D_C $ into $ \il(\textbf{k}) $.
\begin{eqnarray}
	I(\textbf{x}) & = & \sum_\textbf{k} \frac{1}{\Omega_s} e^{i \textbf{k} \cdot \textbf{x}} \il(\textbf{k})
\end{eqnarray}


\subsection{Equation 5}
%Using equation 4, we rewrite equation 3 as
%\begin{eqnarray}
%\langle I(k) I^*(k')\rangle &=&  \sum_\textbf{x} \frac{\Omega_s}{N_\parallel N_\perp} \sum_{\textbf{x}'} \frac{\Omega_s}{N_\parallel N_\perp} \sum_{\textbf{k}''} \frac{P}{\Omega_s D^2 \mathcal{L}}  e^{i \textbf{k}''(\textbf{x} - \textbf{x}')} e^{- i \textbf{k} \textbf{x}} e^{i \textbf{k}' \textbf{x}'} \\
%&=& \sum_\textbf{x} \frac{\Omega_s}{N_\parallel N_\perp} \sum_{\textbf{x}'} \frac{\Omega_s}{N_\parallel N_\perp} \sum_{\textbf{k}''} \frac{P}{\Omega_s D^2 \mathcal{L}}  e^{ -i (\textbf{k} - \textbf{k}'')\textbf{x}} e^{-i (\textbf{k}'' - \textbf{k}') \textbf{x}'} \\
%&=& \sum_{\textbf{k}''} \Omega_s \frac{P}{D^2 \mathcal{L}}  \delta(\textbf{k} - \textbf{k}'')  \delta(\textbf{k}'' - \textbf{k}') \\
%& = & \Omega_s \frac{P}{D^2 \mathcal{L}}  \delta(\textbf{k} - \textbf{k}') \\
%& = & \Omega_s \frac{P}{D^2 \mathcal{L}}  \delta_{ll'} \delta_{kk'}
%\end{eqnarray}
starting with equation 3:
\begin{eqnarray}
	\langle I(k) I^*(k')\rangle  &=& \int d^3x \int d^3 x'  \langle I(\textbf{x}) I'(\textbf{x}') \rangle e^{- i \textbf{k} \textbf{x}} e^{i \textbf{k}' \textbf{x}'}  \\
	&=& \int d^3x \int d^3 x' \int \frac{d^3 k''}{(2\pi)^3} P(\textbf{k}'') e^{i\textbf{k}'' (\textbf{x} - \textbf{x}')} e^{- i \textbf{k} \textbf{x}} e^{i \textbf{k}' \textbf{x}'}  \\
	\langle \tilde{I}(k) \tilde{I}^*(k')\rangle &=&  \sum_\textbf{x} \frac{\Omega_s}{N_\parallel N_\perp} \sum_{\textbf{x}'} \frac{\Omega_s}{N_\parallel N_\perp} \sum_{\textbf{k}''} \frac{P(\textbf{k}'')}{\Omega_s D^2 \mathcal{L}}  e^{i \textbf{k}''(\textbf{x} - \textbf{x}')} e^{- i \textbf{k} \textbf{x}} e^{i \textbf{k}' \textbf{x}'} \\
	&=& \sum_\textbf{x} \frac{\Omega_s}{N_\parallel N_\perp} \sum_{\textbf{x}'} \frac{\Omega_s}{N_\parallel N_\perp} \sum_{\textbf{k}''} \frac{P(\textbf{k}'')}{\Omega_s D^2 \mathcal{L}}  e^{ -i (\textbf{k} - \textbf{k}'')\textbf{x}} e^{-i (\textbf{k}'' - \textbf{k}') \textbf{x}'} \\
	&=&  \Omega_s \frac{P(\textbf{k}'')}{D^2 \mathcal{L}}  \delta^k(\l,\l') \delta^k(j,j')\\
	&=&  \Omega_s C_{\l, j}{D^2 \mathcal{L}}  \delta^k(\l,\l') \delta^k(j,j')
\end{eqnarray}
where $ C_{lj} \equiv \frac{P(\textbf{k}'')}{D^2 \mathcal{L}}$ is the discrete angular power spectrum.


\end{document}

